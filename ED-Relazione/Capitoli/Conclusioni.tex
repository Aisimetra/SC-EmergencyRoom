\chapter{Conclusioni}

In conclusione, date le osservazioni ricavate dalle sperimentazioni effettuate possiamo affermare che il software AnyLogic si dimostra uno strumento utile per supportare la simulazione di scenari che hanno bisogno di essere ottimizzati, valutarne possibili estensioni o semplicemente per rispondere a domande del tipo \textit{“what if…?”}.


La palette di tool disponibili, dalla libreria per la modellazione del movimento di pedoni a quella per la modellazione del movimento di fluidi, lo rende uno strumento versatile ed utilizzabile in moltissimi ambiti. 

La modellazione a blocchi rende questo strumento easy to learn, ma le molte funzionalità e il fatto di essere basato completamente su Java lo rendono hard to master; inoltre la possibilità di integrare algoritmi di learning è interessante, ma non è compresa direttamente nel software sebbene con \textit{Pathmind} sia possibile avere un Helper integrato. 

Nel caso della nostra simulazione è risultato difficile integrare algoritmi di learning: gli agenti che abbiamo modellato seguivano un workflow preciso ed ognuno di essi non perseguiva un obiettivo, ma si muoveva in base al prossimo blocco presente nel flusso. 

Bisogna menzionare che è sempre tuttavia possibile osservare o modificare le variabili a runtime, tramite controlli, per vedere come reagisce il modello. Questo ci permette di avere un controllo maggiore sulla simulazione e di poter effettuare anche degli \textit{“stress test”}.
Molto interessante anche la possibilità di inserire statistiche per le popolazioni di agenti ed osservarne l’andamento tramite grafici, per una miglior comprensione dell'andamento. 

AnyLogic è uno strumento molto ampio e sarebbe necessaria una quantità di tempo più elevata per realizzare e simulare modelli più complessi, ma ciò nonostante si è  presentato pratico e funzionale per la realizzazione del modello scelto.