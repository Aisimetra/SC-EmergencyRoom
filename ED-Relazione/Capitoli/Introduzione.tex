\chapter{Introduzione}

Il tema di questo progetto è basato sostanzialmente sullo studio esplorativo del software AnyLogic, utilizzando un problema del mondo reale come quello di ottimizzazione delle tempistiche di un pronto soccorso.

Siamo quindi andati valutare come questo strumento si comportasse nel modellare questa situazione per valutarne le funzionalità utilizzando un approccio ad agenti. 
All’interno di un pronto soccorso gravitano diverse figure e i pazienti possono arrivare in condizioni molto diverse.

Abbiamo scelto di utilizzare AnyLogic per conoscere un software differente da quelli proposti nel corso e capire se, essendo fornito ed utilizzato a livello commerciale, potesse fornire degli strumenti migliori per rendersi potenzialmente comprensibile ad un gruppo di persone interessate all’argomento ma non ferrate sulla simulazione.

Nei prossimi capitoli di questo documento descriveremo il software scelto, con un approfondimento sul modello ad agenti realizzato e simulato grazie ad esso; infine, analizzeremo i risultati ottenuti da alcune sperimentazioni eseguite, al fine di confrontarli con quello che è lo stato dell'arte attuale, ovvero gli articoli su cui ci siamo basati per fondare il nostro modello. 

All'interno del \href{https://github.com/NArmas-unimib/SC-EmergencyRoom/tree/main}{repository} è possibile scaricare il file della simulazione in AnyLogic e visualizzare i video che ne mostrano il funzionamento.